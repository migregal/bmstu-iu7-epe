\chapter{Лабораторная работа}

\section{Описание проекта}

Команда разработчиков из 16 человек занимается созданием карты города на основе собственного модуля отображения. Проект должен быть завершен в течение 6 месяцев. Бюджет проекта: 50 000 рублей.

\section{Выравнивание загрузки ресурсов в проекте}

Для ликвидации перегрузки ресурсов в проекте будет использоваться автоматическое выравнивание ресурсов.

\includeimage{task-1-0}{f}{h}{0.5\textwidth}{Параметры выравнивания}

Автоматическое выравнивание работает по принципу неизменения критического пути. На рисунке ниже изменившиеся поля отмечены синим цветом.

\clearpage

\includeimage{task-1-1}{f}{h}{0.5\textwidth}{Изменившиеся поля}

Ресурсы более не перегружены.

\includeimage{task-1-2}{f}{h}{0.5\textwidth}{Лист ресурсов}

\clearpage

\includeimage{task-1-3}{f}{h}{0.5\textwidth}{Визуальный оптимизатор ресурсов}

\section{Учет периодических задач в плане проекта}

Создание повторяющейся еженедельной задачи <<Совещание>> по средам с 10 до 11 утра.

\includeimage{task-2-0}{f}{h}{0.5\textwidth}{Назначение ресурсов на задачу <<Совещание>>}

\clearpage

Результат создания повторяющейся еженедельной задачи <<Совещание>>.

\includeimage{task-2-1}{f}{h}{0.5\textwidth}{Стоимость задачи до оптимизации затрат}

На повторяющуюся задачу назначены все специалисты, кроме наборщиков данных и программистов №1--4, т.к. их интересы на совещании представляет ведущий программист.

\includeimage{task-2-2}{f}{h}{0.5\textwidth}{Список ресурсов задачи <<Совещание>>}

После назначения на задачу <<Совещание>> получили перегрузку ресурсов. Стоимость данной задачи составила 20 730 рублей.

Возникшую перегрузку ресурсов можно разреншить посредством автоматического выравнивания и дополнительного прерывания задач, на которых ресурсы остались перегружены. 

Ликвидировать превышение бюджета можно, составив план затрат B для всех ресурсов, назначенных на повторяющуюся задачу <<Совещание>> без учета затрат на использование.

\includeimage{task-2-3}{f}{h}{0.5\textwidth}{Назначение соотвествующим русерсам плана затрат B}

\includeimage{task-2-4}{f}{h}{0.5\textwidth}{Стоимость задачи после оптимизации затрат}

После устранения перегрузки ресурсов и изменения плана затрат стои- мость совещания уменьшилась с 20 730 до 1 769 рублей, а стоимость проекта уменьшилась с 68 085 до 49 845 рублей.

Таким образом, с выравниванием ресурсов после добавления периодической задачи <<Совещание>> дата окончания проекта осталась с 18.09.23, в то время как стоимость увеличилась на 1 769 рублей и составила 49 845 рублей, оставаясь в пределах бюджета.

\section{Оптимизация критического пути}

Среди задач, лежащих на критическом пути, наибольшую длительность имеют задачи, связанные с программированием. Соответственно, эти задачи оказывают наибольшее влияние на срок реализации проекта.

При этом, в визуальном оптимизаторе видно, что программисты распределены по задачам, связанным с программированием, неравномерно.

\includeimage{task-3-0}{f}{h}{0.6\textwidth}{Начальное распределение программистов по их задачам}

После перераспределения программистов по их задачам новых перегрузок не возникло.

\includeimage{task-3-1}{f}{h}{0.6\textwidth}{Итоговое распределение программистов по  задачам}

Исходя из результатов второй лабороторной работы, программисты являются высокооплачиваемыми специалистами, а значит, сокращение времени их работы позволит существенно уменьшить затраты на проект.

По итогам оптимизации дата окончания проекта сдвинулась с 18.09.23 на 28.07.23, а стоимость проекта уменьшилась с 49845 руб. до 48554 руб. (сократилась на 1291 руб.) и не превышает бюджета проекта.

В результате, псле добавления в план проекта совещаний, ликвидации перегрузки ресурсов, оптимизации затрат и критического пути трудозатраты изменились для следующих групп:

\begin{itemize}
    \item[---] сервер --- уменьшились на 2\%;
    \item[---] анализ --- увеличились на 1\%;
    \item[---] ввод данных --- увеличились на 1\%.
\end{itemize}

\includeimage{task-3-2}{f}{h}{0.6\textwidth}{Настройка сведений о проекте}

В то же время, затраты изменились для следующих групп:

\begin{itemize}
     \item[---] сервер --- уменьшились на 1\%;
     \item[---] программирование --- уменьшились на 1\%;
     \item[---] анализ — увеличились на 1\%;
     \item[---] документация — увеличились на 1\%.
\end{itemize}

\includeimage{task-3-3}{f}{h}{0.6\textwidth}{Настройка сведений о проекте}

По результатам выполнения ЛР №3 получились следующие соотношения «Затраты---Трудозатраты»:

\begin{itemize}
    \item[---] анализ = $\frac{11}{3} = 3.67$ --- уменьшилось;
    \item[---] программирование = $\frac{49}{29} = 1.69$ --- уменьшилось;
    \item[---] сервер = $\frac{12}{30} = 0.4$;
    \item[---] ввод данных = $\frac{11}{26} = 0.42$;
    \item[---] документация = $\frac{3}{2} = 1.5$ --- увеличилось;
\end{itemize}

Так же был сохранен бызовый план проекта.

\includeimage{task-3-4}{f}{h}{0.5\textwidth}{Cохраненbt бызового плана проекта}