\chapter{COCOMO}

COnstructive COst MOdel – алгоритмическая модель оценки стоимости разработки программного обеспечения, разработанная Барри Боэмом.
Модель использует простую формулу регрессии с параметрами, определенными из данных, собранных по ряду проектов.

\begin{equation}
\textup{Трудозатраты} = c_1 \times EAF \times \textup{Размер}^{p_1},
\end{equation}

\begin{equation}
\textup{Время} = c_2 \times \textup{Трудозатраты}^{p_2},
\end{equation}

где:

Трудозатраты --- количество человеко-месяцев.

$c_1$ --- масштабирующий коэффициент.

EAF --- уточняющий фактор, характеризующий предметную область, персонал, среду и инструментарий, используемый для создания рабочих продуктов процесса.

Размер --- размер конечного продукта (кода, созданного человеком), измеряемый в исходных инструкциях (DSI, delivered source instructions), которые необходимы для реализации требуемой функциональной возможности.

$p_1$ --- показатель степени, характеризующий экономию при больших масштабах, присущую тому процессу, который используется для создания конечного продукта; в частности, способность процесса избегать непроизводительных видов деятельности (доработок, бюрократических проволочек,
накладных расходов на взаимодействие).

Время --- общее количество месяцев.

$c_2$ --- масштабирующий коэффициент для сроков исполнения.

$p_2$ --- показатель степени, который характеризует инерцию и распараллеливание, присущие управлению разработкой ПО.
