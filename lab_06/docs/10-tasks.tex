\chapter{Лабораторная работа}

\section{Задание 1}

Исследовать зависимость трудоемкости (РМ) и времени разработки (ТМ) от типа проекта (обычный, промежуточный, встроенный) для модели COCOMO. Получить значения PM и ТМ по всем типам проектов, приняв размер программного кода (SIZE) равным 100 KLOC. 

Проанализировать как влияет на трудоемкость и время уровень способностей ключевых членов команды, а также уровень автоматизации среды:
\begin{itemize}
    \item[---] ACAP --- способности аналитика;
    \item[---] PCAP --- способности программиста;
    \item[---] MODP --- использование современных методов;
    \item[---] TOOL --- использование программных инструментов.
\end{itemize}

Для этого получить значения PM и ТМ, изменяя значения указанных драйверов от очень низких до очень высоких. Результаты исследований оформить графически и сделать соответствующие выводы. При необходимости сократить срок выполнения проекта, что повлияет больше: способности аналитика, способности программиста или параметры среды?

Ниже представлены графики, отображающие влияние типа проекта на трудоемкость и время разработки.

\includeimage{task-1-0}{f}{h}{0.7\textwidth}{Влияние типа проекта на трудоемкость и время разработки}

Ниже представлены графики, отображающие влияние атрибутов персонала и автоматизации среды на трудоемкость и время разработки.

\includeimage{task-1-1}{f}{h}{0.7\textwidth}{Влияние атрибутов персонала и автоматизации среды на трудоемкость и время разработки}

Изменение типа проекта от <<Обычного>> до <<Встроенного>> влечет за собой повышение трудозатрат и снижение времени разработки.

Повышение квалификационных характеристик членов команды, а так же автоматизации среды ведет
к снижению трудозатрат и времени разработки. Также стоит отметить, что на сроки реализации наиболее влияющим фактором на высокой сложности проекта оказывается аналитик.

\clearpage

\section{Задание 2}

Компания получила заказ на разработку программного обеспечения для рабочей станции дизайнера автомобиля. Заказчик следующим образом определил проблемную область в своей спецификации: ПО должно формировать 2-х и 3-х мерные изображения для дизайнера, система должна иметь стандартизованный графический интерфейс, геометрические и прикладные данные должны содержаться в базе данных (планируемый размер базы данных не более 200 тыс. записей).  При анализе проекта его размер был предварительно оценен в 140 000 строк кода. Проект реализуется по промежуточному варианту. Все показатели драйверов затрат, кроме трех имеют номинальное значение:
\begin{itemize}
    \item[---] знание языка программирования (LEXP) --- высокая оценка;
    \item[---] использование современных методов (MODP) --- очень высокая оценка;
    \item[---] использование программных инструментов (TOOL) --- низкая оценка (стандартная среда визуального программирования).
\end{itemize}

Произвести оценку показателей проекта по методике СОСОМО.

\clearpage

Был рассчитан проект.

\includeimage{task-2-0}{f}{h}{0.7\textwidth}{Расчет проекта}

Ниже представлена диаграмма привлечения сотрудников.

\includeimage{task-2-1}{f}{h}{0.7\textwidth}{Диаграмма привлечения сотрудников}

Расчет бюджета по данной диаграмме (была взята медианная месячная зарплата за второе полугодие 2022 года по москве по данным сервиса Хабр Карьера):

\begin{itemize}
    \item[---] Менеджер продукта – 230 000 рублей;
    \item[---] Разработчик – 200 000 рублей;
    \item[---] Системный аналитик – 160 000 рублей;
    \item[---] Инженер по тестированию – 150 000 рублей;
    \item[---] ИБ – 139 000 рублей;
    \item[---] Дизайнер – 133 000 рублей.
\end{itemize}

Расчет бюджета в соответствии с планом:
\begin{itemize}
    \item[---] Планирование и определение требований (Менеджер продукта + Системный аналитик + 2x Разработчик + ИБ + Дизайнер) --- 8 496 000 руб.;
    \item[---] Проектирование продукта (Менеджер продукта + 2x Системный аналитик + 7x разработчик + 2x ИБ + Дизайнер) --- 13 344 000 руб.;
    \item[---] Детальное проектирование (5x Менеджер продукта + 5x Системный аналитик + 20x Разработчик + 2x ИБ + 5x Дизайнер) --- 27 572 000 руб.;
    \item[---] Кодирование и тестирование отдельных модулей (30x Разработчик + 9x Инженер по тестированию) --- 29 400 000 руб.;
    \item[---] Интеграция и тестирование (22x Разработчик + 8x Инженер по тестированию) --- 28 800 000 руб.; 
\end{itemize}

Итоговая стоимость проекта: 107 612 000 руб.

Трудоемкость проекта составила 703 человеко-месяца, а время разработки – 32.8 месяца.