\chapter{Задание для тренировки}

\section*{Задание}

\textbf{Вариант}: 3

\begin{enumerate}
    \item Дополнить временной план проекта, подготовленный на предыдущем этапе
(лабораторная работа № 1), информацией о ресурсах и определить стоимость
проекта;
    \item Для этого заполнить ресурсный лист в программе MS Project, принимая во
внимание, что к реализации проекта привлекается не более 10 исполнителей;
    \item Предусмотреть, что стандартная ставка ресурса составляет 200 руб./день.
    \item Произвести назначение ресурсов на задачи в соответствии с таблицей. С учетом
того, что квалификация ресурсов одинаковая, при назначении ресурсов
использовать процент загрузки.
    \item Для выполнения работ С и Е предусмотреть назначение материального ресурса
стоимость 100 рублей за штуку и расходом 2 штуки для работы С и 5 штук для
работы Е.
\end{enumerate}

\begin{table}[!h]
    \begin{center}
        \caption{Временные характеристики проекта}
        \begin{tabular}{|c|c|}
            \hline
            \bfseries Название работы & \bfseries Кол-во исполнителей \\\hline
            Работа A & 5 \\
            Работа B & 7 \\
            Работа C & 1 \\
            Работа D & 3 \\
            Работа E & 2 \\
            Работа F & 3 \\
            Работа G & 6 \\
            Работа H & 1 \\
            Работа I & 5 \\\hline
        \end{tabular}
    \end{center}
\end{table}

\newpage

\section*{Выполнение}

Был заполнен ресурсрный лист (добавлены трудовые и материальные ресурсы).

\includeimage{task-0-0}{f}{h}{1.0\textwidth}{Заполнения листа ресурсов}

Задачам были назначены исполнители.

\includeimage{task-0-1}{f}{h}{1.0\textwidth}{Задачи проекта}

Изменения отображены на диаграмме Ганта.

\includeimage{task-0-2}{f}{h}{1.0\textwidth}{Диаграмма Ганта}

В визуальном оптимизаторе ресурсов можно увидеть наложение задач исполнителей.

\clearpage

\includeimage{task-0-3}{f}{h}{1.0\textwidth}{Наложение задач}
