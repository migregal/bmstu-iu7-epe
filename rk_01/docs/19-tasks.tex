\section{Контроль за реализацией проекта}

Дата отчета установлена на 6.06.23.

Автоматически отмечены проценты выполнения задач к данному сроку.

\includeimage{task-10-0}{f}{h}{0.5\textwidth}{Выполнение задач к дате отчета}

Выведем линию хода выполнения на дату отчета.
\includeimage{task-10-1}{f}{h}{0.5\textwidth}{Задание линии хода выполнения}

\includeimage{task-10-2}{f}{h}{0.5\textwidth}{Линия хода выполнения}

Таким образом видим, что проект выполняется согласно плана.

Чтобы проанализировать влияние отклонения на проект, отведем под задачу <<Программирование сайта 1>> 2 дополнительных дня.

\includeimage{task-10-3}{f}{h}{0.5\textwidth}{Обновление информации о задаче}

\includeimage{task-10-4}{f}{h}{0.5\textwidth}{Линия хода выполнения по завершении внесения изменений}

Видим, что дата завершения проекта сдвинулась, а так же выросли затраты на проект.

\includeimage{task-10-5}{f}{h}{0.5\textwidth}{Таблица освоенного объема}

Таким образом, после внесения отклонений на 06.06:\\
\begin{itemize}
    \item[---] затраты по базому плану --- 706 016 руб.;
    \item[---] отклонения от базового плана:
    \begin{enumerate}
        \item ОКП < 0 --- проект отстает от плана из-за внесения изменений в задачу.
        \item ОПС < 0 --- проект отклоняется от сметы, наблюдается перерасход средств.
        \item ОПЗ < 0 --- имеет место перерасход средств.
    \end{enumerate}
\end{itemize}



