\chapter{Лабораторная работа}

\section{Описание проекта}

Команда разработчиков из 16 человек занимается созданием карты города на основе собственного модуля отображения. Проект должен быть завершен в течение 6 месяцев. Бюджет проекта: 50 000 рублей.

\section{Настройка рабочей среды проекта}

На вкладке \texttt{Проект -> Сведения о проекте} внесены параметры по условию.

\includeimage{task-1-0}{f}{h}{0.6\textwidth}{Настройка сведений о проекте}

На вкладке \texttt{Файл -> Параметры -> Расписание} установлены параметры рабочей недели и планирования.

\includeimage{task-1-1}{f}{h}{0.6\textwidth}{Настройка расписания}

На вкладке \texttt{Проект -> Изменить рабочее время} установлены нерабочие праздничные дни.

\includeimage{task-1-2}{f}{h}{0.6\textwidth}{Настройка нерабочих праздничных дней}

На вкладке \texttt{Задача -> Суммарная задача} установлена суммарная задача проекта и добавлена заметка с основной информацией о проекте.

\includeimage{task-1-3}{f}{h}{0.6\textwidth}{Настройка суммарной задачи}

\clearpage

\section{Создание списка задач}

Осуществлен ввод задач с ручным планированием.

\includeimage{task-2-0}{f}{h}{0.8\textwidth}{Ввод задач}

\section{Структурирование списка задач}

При помощи кнопки \texttt{Понизить уровень задачи} были выделены подзадачи в соответствии с условием.

\includeimage{task-3-0}{f}{h}{0.8\textwidth}{Разбиение на подзадачи}

\clearpage

\section{Установление связей между задачами}

При помощи заполнения колонки \texttt{Предшественник} у каждой задачи были установлены связи между задачами.

\includeimage{task-4-0}{f}{h}{0.8\textwidth}{Установленные связи между задачами}