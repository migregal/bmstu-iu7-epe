\chapter{Задание для тренировки}

\textbf{Вариант}: 3
\textbf{Задание}: Осуществить планирование проекта со следующими временными
характеристиками.

\begin{table}[!h]
    \begin{center}
        \caption{Временные характеристики проекта}
        \begin{tabular}{|c|c|}
            \hline
            \bfseries Название работы & \bfseries Длительность, дни \\\hline
            Работа A & 3 \\
            Работа B & 4 \\
            Работа C & 1 \\
            Работа D & 4 \\
            Работа E & 5 \\
            Работа F & 7 \\
            Работа G & 6 \\
            Работа H & 5 \\
            Работа I & 8 \\\hline
        \end{tabular}
    \end{center}
\end{table}

Дата начала проекта --- 1-ый рабочий день февраля текущего года.
Провести планирование работ проекта, учитывая следующие связи между задачами:
\begin{itemize}
\item Предусмотреть, что D --- исходная работа проекта;
\item Работа E следует за D;
\item Работы A, G и C следуют за E;
\item Работа B следует за A;
\item Работа H следует за G;
\item Работа F следует за C;
\item Работа I начинается после завершения B, H, и F.
\end{itemize}

Использовались стандартные параметры MS Project. Проект продолжался 38 дней. Дата начала --- 1 февраля 2023 года, дата окончания --- 10 марта 2023 года.

\clearpage

\includeimage{task-0-0}{f}{h}{0.6\textwidth}{Решение тренировочного задания}

\includeimage{task-0-1}{f}{h}{0.6\textwidth}{Решение тренировочного задания (диаграмма Ганта)}