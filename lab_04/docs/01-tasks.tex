\chapter{Лабораторная работа}

\section{Описание проекта}

Команда разработчиков из 16 человек занимается созданием карты города на основе собственного модуля отображения. Проект должен быть завершен в течение 6 месяцев. Бюджет проекта: 50 000 рублей.

\section{Индивидуальное задание}

Дата отчета: 12 мая.
Отметить как выполненные все работы, которые должны были завершиться на эту дату, кроме:
\begin{enumerate}
    \item С 20.03 Системный аналитик был на 30\% задействован в другом проекте.
    \item Задача <<Разработка 3D графических элементов>> завершилась 28.03.
    \item С 6.03 зарплата 3D аниматора увеличилась на 15\%.
    \item C 01.04 уволился Программист №1. Загрузка остальных программистов в апреле увеличилась до 9 часов в день с увеличением зарплаты на 20\%. С 10 мая наняли еще одного программиста и продолжительность рабочего дня и зарплаты всех программистов снова стали равны первоначальным.
    \item 3 апреля для задачи «Создание мультимедиа наполнения» купили специализированное ПО стоимостью 600 рублей и еще 100 рублей понадобилось на его установку.
    \item С 1 апреля купили собственный сервер за 3500 р. и отказались от аренды.
\end{enumerate}

\section{Актуализация параметров проекта}

Задана дата отчета.

\includeimage{task-1-0}{f}{h}{0.5\textwidth}{Дата отчета}

Учтена частичная загруженность системного аналитика в другом проекте.

\includeimage{task-1-1}{f}{h}{0.5\textwidth}{Доступность системного аналитика}

Вручную исправлена фактическая дата окончания <<Разработки 3D графических элеентов>>.

\includeimage{task-1-2}{f}{h}{0.5\textwidth}{Изменение задачи}

Увеличена зарплата 3D аниматора.

\includeimage{task-1-3}{f}{h}{0.5\textwidth}{Визуальный оптимизатор ресурсов}

\clearpage

Вручную изменен период доступности ресурса \textit{Программист 1} - сотрудник уволен 01.04, на его место нашли сотрудника 10.05. 

\includeimage{task-1-4}{f}{h}{0.5\textwidth}{Увольнение и принятие на работу программиста}

Для оставшихся программистов изменен режим работы, а так же увеличена ставка на период отсутсвия первого программиста.

\includeimage{task-1-5}{f}{h}{0.5\textwidth}{Введение отдельных рабочих недель для оставшихся программистов}

\clearpage

\includeimage{task-1-6}{f}{h}{0.5\textwidth}{Изменение продолжительности рабочего дня для оставшихся программистов}

\includeimage{task-1-7}{f}{h}{0.5\textwidth}{Изменение ставки для оставшихся программистов}

\clearpage

После приобретения собственного сервера отпала необходимость в аренде. Затраты на покупку отражены в виде фиксированных затрат для задачи.

\includeimage{task-1-8}{f}{h}{0.5\textwidth}{Стоимость аренды сервера}

После выравнивания задач и избавления от перегрузки сотрудников, дата завершения проекта не изменилась, задачи выполняются с опережением сроков.

\includeimage{task-1-9}{f}{h}{0.6\textwidth}{Опережение графика}

По итогам внесения изменений дата окончания проекта сдвинулась с 28.07.23 на 24.08.23 (на 27 дней вперед), а стоимость проекта изменилась с 48554 руб. до 48114 руб. (сократилась на 440 руб.) и не превышает бюджета проекта.

В результате, после добавления в план проекта изменений согласно задания лабораторной работы, трудозатраты изменились для следующих групп:

\includeimage{task-1-10}{f}{h}{0.6\textwidth}{Трудозатраты}

\begin{itemize}
    \item[---] сервер --- увеличились на 1\%;
    \item[---] анализ --- уменьшились на 1\%.
\end{itemize}

В то же время, затраты изменились для следующих групп:

\includeimage{task-1-11}{f}{h}{0.6\textwidth}{Затраты}

\begin{itemize}
     \item[---] сервер --- уменьшились на 12\%;
     \item[---] internet --- увеличились на 1\%;
     \item[---] анализ --- увеличились на 1\%;
     \item[---] ввод данных --- увеличились на 1\%;
     \item[---] дизайн — увеличились на 2\%;
     \item[---] программирование --- увеличились на 7\%;
     \item[---] документация — увеличились на 1\%.
\end{itemize}

