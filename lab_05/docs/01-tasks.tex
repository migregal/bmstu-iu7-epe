\chapter{Лабораторная работа}

\section{Описание проекта}

Команда разработчиков из 16 человек занимается созданием карты города на основе собственного модуля отображения. Проект должен быть завершен в течение 6 месяцев. Бюджет проекта: 50 000 рублей.

\section{Индивидуальное задание}

Дата отчета: 12 мая.
Отметить как выполненные все работы, которые должны были завершиться на эту дату, кроме:
\begin{enumerate}
    \item С 20.03 Системный аналитик был на 30\% задействован в другом проекте.
    \item Задача <<Разработка 3D графических элементов>> завершилась 28.03.
    \item С 6.03 зарплата 3D аниматора увеличилась на 15%.
    \item C 01.04 уволился Программист №1. Загрузка остальных программистов в апреле увеличилась до 9 часов в день с увеличением зарплаты на 20\%. С 10 мая наняли еще одного программиста и продолжительность рабочего дня и зарплаты всех программистов снова стали равны первоначальным.
    \item 3 апреля для задачи «Создание мультимедиа наполнения» купили специализированное ПО стоимостью 600 рублей и еще 100 рублей понадобилось на его установку.
    \item С 1 апреля купили собственный сервер за 3500 р. и отказались от аренды.
\end{enumerate}

\clearpage

\section{Работа с таблицей освоенного объема}

Отображена таблица освоенного объема.

\includeimage{task-1-0}{f}{h}{0.5\textwidth}{Таблица освоенного объема}

Видно, что ОКП отрицательное. Это объясняется запаздыванием проекта на текущую дату.
Значение ОПС положительное, что означает наличие бюджетных запасов на текущий момент.
ОПЗ так же положительное, что означает отсутствие перерасхода. Расход происходит в пределах нормы.

\section{Работа с отчетами проекта}

Создан отчет о бюджетной стоимости.

\includeimage{task-1-1}{f}{h}{0.6\textwidth}{Создание отчета}

\clearpage

\includeimage{task-1-2}{f}{h}{0.6\textwidth}{Созданный отчет}

\includeimage{task-1-3}{f}{h}{0.6\textwidth}{Таблица значений бюджетных затрат}

Отображен понедельный анализ бюджета.

\includeimage{task-1-4}{f}{h}{0.6\textwidth}{Диаграмма значений бюджетных затрат}

Судя по диаграмме наиболее затратным оказался первый квартал, когда шла разработка интерфейсов и ядра системы.

Наиболее затратными оказались 5, 15 и 21 недели, так как в эти недели шли задачи анализа, а так же разработки.

\includeimage{task-1-5}{f}{h}{0.6\textwidth}{Диаграмма превышения затрат}

Затраты превышены по задачам:
\begin{itemize}
    \item[---] создание интерфейса;
    \item[---] создание ядра GIS;
    \item[---] создание мультимедиа-наполнения;
    \item[---] создание web-сайта и поддержка;
\end{itemize}

В тоже время, в рамках задачи <<Построение базы данных объектов>> средства были сэкономлены.

\includeimage{task-1-6}{f}{h}{0.6\textwidth}{Таблица превышения затрат}

\clearpage

\section{Анализ вариантов декомпозиции
работ в проекте}

Декомпозиция по процессам и оптимизация критического пути позволи- ла уменьшить срок проекта (новая дата окончания — 20.07.23), но при этом увеличилась стоисть проекта до 44 954 рублей.

\includeimage{task-1-7}{f}{h}{\textwidth}{Старая декомпозиция}

\includeimage{task-1-8}{f}{h}{\textwidth}{Результаты ЛР №3}

\clearpage

\includeimage{task-1-9}{f}{h}{\textwidth}{Новая декомпозиция}

Сравнивая с результатами ЛР №3, проект завершен на 8 дней раньше, а стоимость проекта в ЛР №3 (48 554,18) больше, чем при декомпозиции (44 954). При этом стоимость совещаний, которых не было в ЛР №2, составила 1342 рубля. 

Т.е. даже без учета совещаний разница составила 2258 рублей. Т.е. удалось достичь сокращения стоимости реализации проекта и более раннего завершения, в сравнении с результатами ЛР №3.