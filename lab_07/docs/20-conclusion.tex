\chapter{Выводы}

Методика COCOMO II, как и ее первая версия, позволяет досочно быстро оценить длительность и трудозатраты проекта, основываясь на субъективных данных. В условиях отсутствия объективной информации о предполагаемых трудозатратах особенно важно правильно спрогнозировать характеристики проекта. 

Методика функциональных точек позволяет оценить размер программного продукта на этапе его проектирования. С помощью этого подхода можно применять методики COCOMO, которым необходимо знание о размере продукта. Хотя методика функциональных точек является неточной (как и любая методика прогнозирования), этих результатов достаточно для общего понимания объемов работ и получения приблизительных оценок параметров проекта.

В ходе выполнения лабораторной работы было выяснено, что модель композиции приложения дает менее оптимистичный прогноз, по сравнению с моделью ранней архитектуры приложения.
