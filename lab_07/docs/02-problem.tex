\chapter{Проект}

Компания получила заказ на разработку автоматизированной информационной системы оплаты штрафов ГИБДД. Оплата штрафов возможна через веб-интерфейс веб-портала и через приложение для мобильного телефона.

\section{Роли пользователей}

В системе предусмотрено два вида пользователей: Пользователь (User) и Администратор (Administrator). Пользователь может просматривать и оплачивать штрафы, 

Администратору доступен просмотр всех выплат, просмотр данных пользователей, их редактирование и создание новых пользователей.

Все записи типа «Пользователь» в системе имеют следующие поля: id, логин, пароль, тип, регистрационный номер водительского удостоверения, номер банковской карты.

\section{Модули}
Информационная система состоит из следующих модулей:

\begin{itemize}
\item[---] приложение для мобильного телефона;
\item[---] веб-портал;
\item[---] модуль регистрации и авторизации;
\item[---] модуль обмена данными с системой ГИБДД;
\item[---] модуль проведения платежных транзакций.
\end{itemize}

\subsection{Приложение для мобильного телефона}

Содержимое:
\begin{itemize}
    \item[---] страница регистрации, где пользователь заполняет следующие поля: логин, пароль, номер водительского удостоверения, номер банковской карты;

    \item[---] страницу просмотра штрафов, на которой отображаются неоплаченные штрафы, и есть кнопка “оплатить” для каждого штрафа. После нажатия кнопки пользователю приходит ответ о положительном или отрицательном результате выполнения операции.
\end{itemize}

\subsection{Веб-портал}

Обеспечивает те же функциональные возможности для Пользователя и в дополнение к этому он имеет панель Администратора. 

\subsection{Модуль регистрации и авторизации}

Позволяет добавлять пользователей в
базу данных.

\subsection{Модуль обмена данными с системой ГИБДД}

Предназначен для получения списка штрафов, а также оповещения ГИБДД об оплате штрафа. При отправке сообщения с номером водительского удостоверения информационной системе ГИБДД модуль обмена данными получает список
штрафов. 

При этом каждый штраф описывается следующими полями: номер постановления, дата постановления, имя, фамилия, отчество нарушителя, сумма штрафа. 

На сообщение об оплате система ГИБДД присылает подтверждение об удаления штрафа из списка неоплаченных или отказ.

\textbf{Примечание}: Система не хранит штрафы в своей базе данных, а получает их из системы ГИБДД при каждом запросе.

\subsection{Модуль проведения платежных транзакций}

Модуль по защищенному протоколу отправляет платежной системе запрос с указанием номера карты пользователя, номера счета ГИБДД и суммы на выполнение проведения оплаты. 

Система отсылает положительный или отрицательный ответ о результате выполнения. В отличие от штрафов все операции по оплате сохраняются в базу данных.

\section{Характеристики команды, продукта и проекта}

\begin{itemize}
\item[---] обмен данными --- 5;
\item[---] распределенная обработка --- 5;
\item[---] производительность --- 3;
\item[---] эксплуатационные ограничения по аппаратным ресурсам --- 0;
\item[---] транзакционная нагрузка --- 3;
\item[---] интенсивность взаимодействия с пользователем (оперативный ввод
данных) --- 2;
\item[---] эргономические характеристики, влияющие на эффективность работы
конечных пользователей --- 0;
\item[---] оперативное обновление --- 4;
\item[---] сложность обработки --- 4;
\item[---] повторное использование --- 3;
\item[---] легкость инсталляции --- 0;
\item[---] легкость эксплуатации/администрирования --- 3;
\item[---] портируемость --- 5;
\item[---] гибкость --- 0.
\end{itemize}

При разработке ПО 30 \% кода будет написано на SQL, 10 \% --- на JavaScript, 60 \% --- на Java. 

К разработке проекта планируется привлечь довольно слаженную команду высокопрофессиональных разработчиков, у которых, однако, практически отсутствует опыт в разработке систем подобного типа. При этом заказчик настаивает на довольно строгом процессе с периодической демонстрацией рабочих продуктов, соответствующих этапам жизненного цикла. Учитывая новизну проекта для команды, на этапе его подготовки был осуществлен относительно детальный анализ рисков, свойственных архитектуре разрабатываемой системы. Организация находится чуть выше второго уровня зрелости процессов разработки.

Надежность и уровень сложности (RCPX) разрабатываемой системы оцениваются как очень высокие, проект не предусматривает специальных усилий на повторное использование компонентов (RUSE). Возможности персонала (PERS) можно охарактеризовать как очень высокие, однако опыт членов команды в данной сфере (PREX) является скорее низким. Сложность платформы (PDIF) средняя. Разработка предусматривает интенсивное использование инструментальных средств поддержки (FCIL). Учитывая новизну проекта для команды и высокие требования по надежности, Заказчик не настаивает на жестком графике (SCED).